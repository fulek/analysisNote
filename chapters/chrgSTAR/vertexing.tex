\section{Vertex Reconstruction}\label{section:star_vertex}

In $pp$ collisions, where the charged-particle multiplicity is low, the vertex finding algorithm sometimes fails to find the~primary vertex. In addition, at high luminosity, vertex finder can fail due to the contribution of pile-up events and providing a wrong reconstructed vertex. In the study of vertex reconstruction efficiency we required at least two reconstructed global tracks $n^\textrm{global}_\textrm{sel}\geq 2$ passing all the quality cuts listed in Sec~\ref{section:star_track_selection} but without $\textrm{DCA}_{xy}$ and $\textrm{DCA}_{z}$ cuts and  associated to a true-level primary particles.  Additionally, MC events were accepted if the~$z$-coordinate of the~true-level primary vertex was between $-80$ and $80$~cm and $n_\textrm{ch}\geq 2$. All corrections, described in this section, were calculated in three ranges of $\xi$ separately using PYTHIA~8 \ac{SD} embedding \ac{MC}.

%\subsubsection{Track Quality Cuts Used for Vertexing}
The global tracks (not necessarily associated to a~true-level primary particles), which are used by the~vertex-finder algorithm, had to pass the~following quality cuts:
%The following quality cuts had to be passed by the global tracks (not necessarily associated to a~true-level primary particles) used by the~vertex-finder algorithm:% to used in the vertex reconstruction:
\begin{enumerate}
	\item Tracks must be matched with hits reconstructed in TOF,
	\item The number of the TPC hits used in the helix fit $N_\textrm{hits}^\textrm{fit}$ must be greater than 20,
	\item The ratio of the number of TPC hits used in the helix fit to the number of possible TPC hits $N_\textrm{hits}^\textrm{fit}/N_\textrm{hits}^\textrm{poss}$ must be greater than $0.52$,
	\item The transverse impact parameter with respect to the beamline $d_0$ must be less than 2 cm,
	\item The track's transverse momentum $p_\textrm{T}$ must be greater than $0.2$~GeV/c.
\end{enumerate}
 The~above track selection criteria are different than those used in the analysis. Thus, primary vertex reconstruction efficiency and fake vertex rate were calculated as a function of the~number of global tracks used in vertexing $n^\textrm{global}_\textrm{vrt}$ instead of $n^\textrm{global}_\textrm{sel}$. 

%\subsubsection{Vertex Efficiency and Fake Vertex Rate}
In the analysis exactly one vertex with $n_\textrm{sel}\geq 2$ is required.  However, in the~study of vertex reconstruction, events with additional vertices were allowed. Therefore, we define the~best vertex as the~reconstructed vertex with the highest number of TOF-matched tracks. This vertex does not have to be associated to true-level primary vertex. The algorithm, which matches reconstructed vertices to true-level vertices,  checks for reconstructed tracks originating from them. If at least one reconstructed track  is assigned to a true-level particle, then the reconstructed vertex is assigned to the true-level vertex from which the true-level particle originates. Since the fake vertices (not matched to the true-level primary vertex) are allowed in the analysis, the overall vertex-finding efficiency, $\epsilon_\textrm{vrt}\left(n_\textrm{vrt}^\textrm{global}\right)$, is expressed as:
\begin{equation}
\epsilon_\textrm{vrt}\left(n_\textrm{vrt}^\textrm{global}\right)=\epsilon_\textrm{vrt}^\textrm{best}\left(n_\textrm{vrt}^\textrm{global}\right)+\delta_\textrm{vrt}^\textrm{fake}\left(n_\textrm{vrt}^\textrm{global}\right)
\end{equation}
where:
\begin{description}
	\item $\epsilon_\textrm{vrt}^\textrm{best}\left(n_\textrm{vrt}^\textrm{global}\right)$ is the primary vertex reconstruction efficiency, determined as the ratio of the~number of good reconstructed events (reconstructed best primary vertex with $n_\textrm{sel}\geq 2$ and matched to the~true-level primary vertex) to the number of input MC events,
	\item $\delta_\textrm{vrt}^\textrm{fake}\left(n_\textrm{vrt}^\textrm{global}\right)$ is the fake vertex rate, determined as the ratio of the number of good reconstructed events (reconstructed best primary vertex with $n_\textrm{sel}\geq 2$ and not matched to the true-level primary vertex) to the number of input MC events. Due to the~contribution of pile-up, it is possible that the~best vertex originates from fake tracks instead of true-level particles.
\end{description}

The vertex-finding efficiency as a function of $n^\textrm{global}_\textrm{vrt}$ is shown in  Fig.~\ref{fig:vertexEffi}~(left). When there are exactly two global tracks used in the vertex reconstruction, $n^\textrm{global}_\textrm{vrt}=2$, the vertex-finding efficiency depends on the longitudinal distance between these tracks $|\Delta z_0|$. Therefore, the vertex finding efficiency for such events $\epsilon_\textrm{vrt}\left(|\Delta z_0|\right)$ is given by:
\begin{equation}
\epsilon_\textrm{vrt}\left(|\Delta z_0|\right)=\epsilon_\textrm{vrt}^\textrm{best}\left(|\Delta z_0|\right)+\delta_\textrm{vrt}^\textrm{fake}\left(|\Delta z_0|\right)
\end{equation}
where: $\epsilon_\textrm{vrt}^\textrm{best}\left(|\Delta z_0|\right)$ is the primary vertex reconstruction efficiency, $\delta_\textrm{vrt}^\textrm{fake}\left(|\Delta z_0|\right)$ is the fake vertex rate.

Figure~\ref{fig:vertexEffi}~(right) shows the vertex finding efficiency for events with $n^\textrm{global}_\textrm{vrt}=2$. This efficiency is smaller than $20\%$ for tracks with $|\Delta z_0|>2$~cm, hence the analysis was limited to  events with  $|\Delta z_0|<2$~cm, when $n^\textrm{global}_\textrm{vrt}=2$. 
\begin{figure}[h!]
	\centering
		\includegraphics[width=0.49\textwidth,page=1]{chapters/chrgSTAR/img/vertex/vertexEffi_ksi.pdf}
		\includegraphics[width=0.49\textwidth,page=8]{chapters/chrgSTAR/img/vertex/vertexEffi_ksi.pdf}
		\caption{Vertex-finding efficiency in three ranges of $\xi$ as a function of  (left) $n^\textrm{global}_\textrm{vrt}$ and (right) with respect to the $|\Delta z_0|$ between reconstructed tracks in events with $n^\textrm{global}_\textrm{vrt}=2$. }
		\label{fig:vertexEffi}
\end{figure}

%\subsubsection{Other Corrections to the Reconstructed Vertices}
%Events with reconstructed best vertex are rejected if there  are:
Events are rejected if additional vertices are reconstructed in addition to the~best one. Rejected events can be classified as:

\begin{enumerate}[label=\alph*)]
	\item two or more additional vertices,
	\item additional  secondary vertex from interactions with the detector dead-material,
	\item additional fake  vertex,
	\item additional primary  vertex (vertex splitting or background vertex reconstructed as best vertex),
	\item additional decay TOF vertex.  
\end{enumerate}
The correction for vetoing such events, $\epsilon_\textrm{vrt}^\textrm{veto}\left(n_\textrm{vrt}^\textrm{global}\right)$, is given by: 
\begin{equation}
\begin{split}
\epsilon_\textrm{vrt}^\textrm{veto}\left(n_\textrm{vrt}^\textrm{global}\right) & =1-\frac{\textrm{number of events with more than one reconstructed  TOF vertex}}{\textrm{number of events with at least one reconstructed TOF vertex}} \\
& =1-\epsilon_a-\epsilon_b-\epsilon_c-\epsilon_d-\epsilon_e
\end{split}
\label{eq:vertexVetoEq}
\end{equation}
where $\epsilon_a-\epsilon_e$ are the fractions of events with additional vertices, whose labels  are listed above.% shown in \ref{fig:vertexVeto}.

As before, the correction was calculated as a function of $|\Delta z_0|$ for events with $n^\textrm{global}_\textrm{vrt}=2$. Figure~ \ref{fig:vertexVeto} shows the fraction of multi-vertex events  with respect to the $n_\textrm{vrt}^\textrm{global}$. There is a~large fraction of events ($>90\%$) with additional background vertices for $n_\textrm{vrt}^\textrm{global}\geq 9$, what would result in large correction factor. Hence, the analysis was limited to events with $n_\textrm{sel}\leq8$. The total fraction of multi-vertex events, $\epsilon_a+\epsilon_b+\epsilon_c+\epsilon_d+\epsilon_e$, as a function of $n^\textrm{global}_\textrm{vrt}$ and $|\Delta z_0|$, shown in Fig.~\ref{fig:vertexVetoDZ}, demonstrates that $\epsilon_\textrm{vrt}^\textrm{veto}(|\Delta z_0|)$ is very large ($>98\%$) for events with $n^\textrm{global}_\textrm{vrt}=2$.
\begin{figure}[h!]
	%\vspace{-0.5cm}
	\centering
	\begin{subfigure}{.47\textwidth}
		\includegraphics[width=\textwidth,page=2]{chapters/chrgSTAR/img/vertex/vertexEffi_ksi.pdf}
	\end{subfigure}
	\begin{subfigure}{.47\textwidth}
		\includegraphics[width=\textwidth,page=9]{chapters/chrgSTAR/img/vertex/vertexEffi_ksi.pdf}
	\end{subfigure}
	\caption{Total fraction of multi-vertex events as a function of (left) $n_\textrm{vrt}^\textrm{global}$ for events with $n^\textrm{global}_\textrm{vrt}>2$ and (right) $|\Delta z_0|$ for events with $n^\textrm{global}_\textrm{vrt}=2$  in three ranges of $\xi$.}
		\label{fig:vertexVetoDZ}
	%\vspace{-0.5cm}
\end{figure}
\begin{figure}[h!]
	%\vspace{-0.5cm}
	\centering
	\begin{subfigure}{.47\textwidth}
		\includegraphics[width=\textwidth,page=3]{chapters/chrgSTAR/img/vertex/vertexEffi_ksi.pdf}
	\end{subfigure}
	\begin{subfigure}{.47\textwidth}
		\includegraphics[width=\textwidth,page=4]{chapters/chrgSTAR/img/vertex/vertexEffi_ksi.pdf}
	\end{subfigure}
	\begin{subfigure}{.47\textwidth}
		\includegraphics[width=\textwidth,page=5]{chapters/chrgSTAR/img/vertex/vertexEffi_ksi.pdf}
	\end{subfigure}
	\begin{subfigure}{.47\textwidth}
		\includegraphics[width=\textwidth,page=6]{chapters/chrgSTAR/img/vertex/vertexEffi_ksi.pdf}
	\end{subfigure}
	\begin{subfigure}{.47\textwidth}
		\includegraphics[width=\textwidth,page=7]{chapters/chrgSTAR/img/vertex/vertexEffi_ksi.pdf}
	\end{subfigure}
	\begin{minipage}{.47\textwidth}
			\caption{Fraction of multi-vertex events  with respect to the $n_\textrm{vrt}^\textrm{global}$ in three ranges of $\xi$. Each contribution is shown separately: (top left) more than one additional vertices, (top right) additional secondary vertex from the interactions with the detector dead-material, (middle left) additional fake vertex, (middle right) additional primary vertex and (bottom) additional decay vertex.}
			\label{fig:vertexVeto}
	\end{minipage}
	%\vspace{-1.5cm}
\end{figure}

Although, the analysis was limited to $n_\textrm{sel}\leq8$, a~fraction  of events  with additional background vertices was still relatively large.
Since most of these additional vertices are fake, it was checked whether the~charged-particle multiplicity distributions are different for events with and without reconstructed fake vertices. These distributions, as shown in Fig~\ref{fig:nchVertex}, are in good agreement, thus, above studies of vertex reconstruction were repeated using \ac{MC} events that do not contain reconstructed fake vertices. The~vertex finding efficiency, which was  calculated from such events, is shown in Fig.~\ref{fig:vertexEffi_noFake}. It is greater than $95\%$ for events with $2 \leq n_\textrm{vrt}^\textrm{global}\leq 8$. 
  In addition, the~corresponding fraction of multi-vertex events, shown in \cref{fig:vertexVetoDZ_noFake,fig:vertexVeto_noFake}, is smaller than $20\%$. Since fake vertices were rejected from this study, the~$\epsilon_{c}$ term from Eq.~\eqref{eq:vertexVetoEq} is equal to $0$. 
  
  The correction factors calculated from \ac{MC} events that do not contain reconstructed fake vertices   are relatively small, hence, they were used in the analysis instead of the~one obtained from the~full \ac{MC} sample.

\begin{figure}[h!]
	%\vspace{-0.5cm}
	\centering
	\begin{subfigure}{.47\textwidth}
		\includegraphics[width=\textwidth,page=1]{chapters/chrgSTAR/img/vertex/nchFake.pdf}
	\end{subfigure}
	\begin{minipage}{.47\textwidth}
		\caption{Charged-particle mulitplicity distributions in three ranges of $\xi$ calculated from PYTHIA~8 SD embedding  MC for (full points) all generated events and (open points) events without reconstructed fake vertices.}
		\label{fig:nchVertex}
	\end{minipage}
	\vspace{-0.5cm}
\end{figure}

\begin{figure}[h!]
	\centering
	\includegraphics[width=0.49\textwidth,page=1]{chapters/chrgSTAR/img/vertex/vertexEffi_ksi_noFake.pdf}
	\includegraphics[width=0.49\textwidth,page=8]{chapters/chrgSTAR/img/vertex/vertexEffi_ksi_noFake.pdf}
	\caption{Vertex-finding efficiency in three ranges of $\xi$ as a~function of  (left) $n^\textrm{global}_\textrm{vrt}$ and (right) with respect to the $|\Delta z_0|$ between reconstructed tracks in events with $n^\textrm{global}_\textrm{vrt}=2$. Only events that do not contain fake vertices were used. }
	\label{fig:vertexEffi_noFake}
\end{figure}

\begin{figure}[h!]
	%\vspace{-0.5cm}
	\centering
	\begin{subfigure}{.47\textwidth}
		\includegraphics[width=\textwidth,page=2]{chapters/chrgSTAR/img/vertex/vertexEffi_ksi_noFake.pdf}
	\end{subfigure}
	\begin{subfigure}{.47\textwidth}
		\includegraphics[width=\textwidth,page=9]{chapters/chrgSTAR/img/vertex/vertexEffi_ksi_noFake.pdf}
	\end{subfigure}
	\caption{Total fraction of multi-vertex events as a function of (left) $n_\textrm{vrt}^\textrm{global}$ for events with $n^\textrm{global}_\textrm{vrt}>2$ and (right) $|\Delta z_0|$ for events with $n^\textrm{global}_\textrm{vrt}=2$  in three ranges of $\xi$. Only events that do not contain fake vertices were used. }
	\label{fig:vertexVetoDZ_noFake}
	%\vspace{-0.5cm}
\end{figure}

\begin{figure}[h!]
	%\vspace{-0.5cm}
	\centering
	\begin{subfigure}{.47\textwidth}
		\includegraphics[width=\textwidth,page=3]{chapters/chrgSTAR/img/vertex/vertexEffi_ksi_noFake.pdf}
	\end{subfigure}
	\begin{subfigure}{.47\textwidth}
		\includegraphics[width=\textwidth,page=4]{chapters/chrgSTAR/img/vertex/vertexEffi_ksi_noFake.pdf}
	\end{subfigure}
	\begin{subfigure}{.47\textwidth}
		\includegraphics[width=\textwidth,page=6]{chapters/chrgSTAR/img/vertex/vertexEffi_ksi_noFake.pdf}
	\end{subfigure}
	\begin{subfigure}{.47\textwidth}
		\includegraphics[width=\textwidth,page=7]{chapters/chrgSTAR/img/vertex/vertexEffi_ksi_noFake.pdf}
	\end{subfigure}
	\caption{Fraction of multi-vertex events  with respect to the $n_\textrm{vrt}^\textrm{global}$ in three ranges of $\xi$. Each contribution is shown separately: (top left) more than one additional vertices, (top right) additional secondary vertex from the interactions with the detector dead-material, (bottom left) additional primary vertex and (bottom right)  additional decay vertex. Only events that do not contain fake vertices were used.}
		\label{fig:vertexVeto_noFake}
	%\vspace{-1.5cm}
\end{figure}

\FloatBarrier
