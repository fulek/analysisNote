\section{Correction to Transverse Momentum and Pseudorapidity Distributions}\label{section:star_dNdeta_dNdpt}
First the accidental and non-SD backgrounds were subtracted from the $p_\textrm{T}$ and $\bar{\eta}$ distributions. Next, each event was  corrected for vertex reconstruction efficiency by applying $w_\textrm{ev}^\textrm{vrt}(n_\textrm{vrt}^\textrm{global},|\Delta z_0|)$ weights. Then, the tracks were corrected for the track reconstruction efficiency, non-primary track background contribution, track and $\xi$ migrations, BBC-small efficiency (the product of $w_\textrm{trk}(p_\textrm{T},\eta,V_z)$, $f_\xi$ and $w_\textrm{BBC}$ weights was applied, $f_\xi$ and $w_\textrm{BBC}$ were calculated as a~function of true-level $p_\textrm{T}$ and $\bar{\eta}$ separately). 


In order to obtain  charged-particle densities, the~$p_\textrm{T}$ and $\bar{\eta}$ distributions   were normalized to unity and scaled by the average charged particle multiplicity in an event $\langle n_\textrm{ch}\rangle$. The~latter was calculated from the corrected charged particle multiplicity distribution $dN/dn_\textrm{ch}$ (Sec.~\ref{section:star_dNdnch}).
The~above procedure was done to  correct  the~data also for events that are lost due to $n_\textrm{sel}<2$ but $n_\textrm{ch}\geq 2$ since such correction  was not included in any event-by-event and track-by-track weights. There was an~assumption that $p_\text{T}$ and $\eta$ distributions are the~same for lost and measured events, but it was validated by the~closure tests~(Sec.~\ref{section:star_closure}).
The~mean $p_\textrm{T}$ and $\bar{\eta}$ in an event, $\langle p_\textrm{T}\rangle$ and $\langle \bar{\eta}\rangle$, were obtained from the~measured distributions.