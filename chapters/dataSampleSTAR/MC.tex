\section{Monte Carlo Samples }\label{section:star_mc}
The factors used to correct the data for the detector effects were obtained by the embedding MC technique~\cite{STAR:tpc}, where  simulated particles are mixed with the real Zerobias events at the raw data level. Zerobias data events used in the embedding were sampled over the entire data-taking period in order to properly describe the data set used in the analysis.  Two samples of embedding MC were produced:
\begin{enumerate}
	\item Single particle MC, where  particle kinematics is taken from flat distributions in $\eta$ and $p_T$. The flat  distributions were used in order to have similar statistics in all bins.
	\item The SaS (Sch{\"u}ler and Sj{\"o}strand)~\cite{PYTHIA:SaS} model implemented in PYTHIA8~\cite{PYTHIA8:Intro} with 4C. 
\end{enumerate}
The particles were propagated through the full simulation of the STAR-TPC and RP system detectors using GEANT3~\cite{GEANT:three} and GEANT4~\cite{GEANT:three}, respectively. The obtained information for the simulated particles was blended into the existing information of the real data. Next, these events were processed through the  reconstruction chain. 

It is preferred to get the detector true-level corrections from the MC, which is dedicated to the studied  physics process. However, for this purpose, the statistics in the MC should be several times greater than we have in the data for analysis. Since this is not possible with  low efficiency of TPC and TOF, the basic method of corrections used in the analysis is the method of factorization of global efficiency into the product of single-particle efficiencies. Thereby, statistically precise multidimensional corrections on TPC and TOF are obtained from the single particle MC.

Additionally, a several pure MC samples were generated. The simulated particles were propagated through full simulation and reconstruction chain but were  not embedded  into Zerobias events.   
Systematic effects related to hadronization of the diffractive system was determined by an alternative model implemented in HERWIG~\cite{Herwig:intro}. The comparison to the corrected data distribution was done for these two generators, in addition all results were compared to the EPOS~\cite{EPOS:intro} and alternative PYTHIA 8 model  MBR (Minimum Bias Rockefeller)~\cite{MBR:intro}  with A2 tune~\cite{ATLAS:A2}. EPOS predicts very large contribution of forward 
protons well separated in rapidity from other final state particles from non-diffractive events. This is the result of low mass excitation of the proton remnant ($<1$ GeV) leading to hadronization of the beam remnant back to the proton. Therefore for the comparison with data EPOS predictions were separated in two classes: diffractive (EPOS-SD) modelled by Pomeron exchange and non-diffractive modelled  by low mass excitation of the proton remnant (EPOS-SD$^\prime$).
In all PYTHIA 8 models diffractive cross sections are arbitrary suppressed at 
relatively large values of $\xi$ ($>$0.05). This arbitrary suppression significantly changes predicted distribution of $\xi$ and fractions of different processes in our fiducial phase space. Therefore data was also compared with expectations obtained without suppression of diffractive cross sections (MBR-tuned).