\section{Event Selection}\label{section:star_event_selection}
Events were selected from those triggering the SDT trigger condition. In order to remove events having poor quality the following cuts were applied:
\begin{enumerate}
	\item RP trigger in exactly two stations of one arm,
	\item Any signal in small BBC tiles or ZDC on the opposite side of the STAR central detector to the triggered RP station,
	\item Exactly one proton track in the above RP stations with $0.02 < \xi < 0.2$ and $0.04 < -t < 0.16$~GeV$^{2}$/c$^{2}$. 
	\item Exactly one primary vertex with TPC tracks matched with hits in TOF (later in the text such vertex  is refered as a \textit{TOF vertex}),
	\item TPC vertex is placed within $|V_z|<80$~cm - events with vertices away from the IP have low acceptance for the central and forward tracks,
	\item At least two but no more than eight primary TPC tracks, $2\leq n_{sel}\leq 8$, matched with hits in TOF and satisfying the selection criteria described in Sec.~\ref{section:star_track_selection},
	\item If there are exactly two primary tracks satisfying above criteria and exactly two global tracks used in vertex reconstruction (Sec.~\ref{section:star_vertex}), the longitudinal distance between these global tracks should be smaller than $2$~cm, $|\Delta z_0|<2$~cm.
\end{enumerate}
Figure~\ref{fig:vertexEffi} shows the multiplicity of TOF vertices (left) and the $z$-position of primary vertex in a single TOF vertex events (right). 
\begin{figure}[h!]
	\centering
	\includegraphics[width=.49\textwidth, page=10]{chapters/chrgSTAR/img/selection/SDT.pdf}
	\includegraphics[width=.49\textwidth, page=5]{chapters/chrgSTAR/img/selection/SDT.pdf}
	\caption[Primary vertex multiplicity and the $z$-position of primary vertex in a single TOF vertex events  before applying  the corresponding cuts]{Primary vertex multiplicity (left) and the $z$-position of primary vertex in a single TOF vertex events (right) before applying  the corresponding cuts. Blue lines indicate regions accepted in the analysis.}
	\label{fig:vertexSTAR}
\end{figure}

\FloatBarrier