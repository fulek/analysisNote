\section{Fiducial Region of the Measurement}\label{section:star_fiducial}
A fiducial phase space of measurement  is defined by the~following criteria. Primary charged particles are defined as charged particles with a mean lifetime $\tau >300$~ps, either directly produced in $pp$ interaction or from subsequent decays of directly produced particles with $\tau <30$~ps. In this analysis the total number of primary charged particles (without identification), $n_\textrm{ch}$, was required to be  $2\leq n_\textrm{ch} \leq 8$. These primary charged particles had to be contained within the kinematic range of $p_\textrm{T}>0.2$~GeV/c and $|\eta|<0.7$. In identified charged antiparticle to particle ratio measurement, the lower transverse momentum limit was changed for the analyzed particles as follows: $0.2$~GeV/c (pions), $0.3$~GeV/c (kaons), $0.4$~GeV/c (protons and antiprotons)

The measurements were performed in a fiducial phase space of the forward scattered protons of $0.04<-t<0.16$~GeV$^{2}$/c$^2$ and $0.02 < \xi<0.2$. All measured observables are presented in three $\xi$ regions: $0.02<\xi<0.05$, $0.05<\xi<0.1$ and $0.1<\xi<0.2$.
\FloatBarrier