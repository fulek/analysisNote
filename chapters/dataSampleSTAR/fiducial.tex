\section{Fiducial Region of the Measurement}\label{section:star_fiducial}
A fiducial region of measurement that is compatible with above selection was chosen. Primary charged particles are defined as charged particles with a mean lifetime $\tau >300$~ps, either directly produced in $pp$ interactions or from subsequent decays of directly produced particles with $\tau <30$~ps. In this analysis the total number of primary charged particles (without identification), $n_{ch}$, was required to be between two and eight, $2\leq n_{ch} \leq 8$. These primary charged particles had to be contained within the kinematic range of $p_T>0.2$~GeV/c and $|\eta|<0.7$. In identified charged antiparticle to particle ratios measurement, the lower transverse momentum limit was changed for the analyzed particles as follows: 
\begin{table}[h!]
	\begin{tabular}{ll}
	pions: & $p_T>0.2$~GeV/c\\
	kaons: & $p_T>0.3$~GeV/c\\
	(anti)protons: & $p_T>0.4$~GeV/c\\
	\end{tabular}
\end{table}

The measurements were performed in a fiducial phase space of the forward scattered protons of $0.04<-t<0.16$~GeV$^{2}$/c$^2$ and $0.02 < \xi<0.2$. All measured observables are presented in three $\xi$ regions: $0.02<\xi<0.05$, $0.05<\xi<0.1$ and $0.1<\xi<0.2$.
\FloatBarrier