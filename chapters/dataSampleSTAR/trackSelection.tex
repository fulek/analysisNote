\section{Track Selection}\label{section:star_track_selection}
The following quality cuts had to be passed by the selected primary tracks in this analysis:
\begin{enumerate}
	\item The tracks must be matched with hits reconstructed in TOF,
	\item The number of the  TPC hits used in the helix fit $N_{hits}^{fit}$ must be greater than $24$,
	\item The number of the  TPC hits used to determine the $dE/dx$ information $N_{hits}^{dE/dx}$ must be greater than $14$,
	\item The transverse impact parameter with respect to the beamline $d_0$ must be less than $1.5$~cm,
	\item The radial component of the distance of the closest approach between  the global helix and the vertex $DCA_{xy}$ must be less than $1.5$~cm (consistent with the $d_0$ limit),
	\item The absolute magnitude of  longitudinal component of the distance of the closest approach between  the global helix and the vertex $|DCA_{x}|$ must be less than $1$~cm,
	\item The track's transverse momentum $p_T$ must be greater than $0.2$~GeV/c,
	\item The track's absolute value of  pseudorapidity $|\eta|$ must be smaller than $0.7$.
\end{enumerate}
The $N_{hits}^{fit}$ cut is used to reject low quality TPC tracks and avoid track splitting effects. The $d_0$ and global $DCA_{xy}$,  $|DCA_{z}|$ cuts are used to select tracks that originate from the primary interaction vertex. The cut on $N_{hits}^{dE/dx}$ is used to ensure that selected tracks have sufficient energy loss information
for particle identification purposes. In this analysis tracks without identification are required to have $p_T > 0.2$~GeV/c and $|\eta| < 0.7$ due to high track reconstruction and TOF matching efficiencies in that region. For the identified particle-antiparticle ratio analysis, where in addition to charged pions, charged kaons and (anti)proton  are measured, the $p_T$ cut was increased  to $0.3$ and $0.4$~GeV/c, respectively. The full $2\pi$ azimuthal coverage of the TPC is utilized.
The effects of $DCA_{xy}$, $|DCA_{z}|$, $d_0$, $N_{hits}^{fit}$ and $N_{hits}^{dE/dx}$ cuts are shown in Fig.~\ref{fig:dca_nhitsSTAR}. 
\captionsetup{format=plain,indention=0pt,justification=justified}
\begin{figure}[h!]
	\centering
	\begin{subfigure}{.45\textwidth}
	\includegraphics[width=\textwidth, page=8]{chapters/chrgSTAR/img/selection/SDT.pdf}
	\end{subfigure}
	\begin{subfigure}{.45\textwidth}
	\includegraphics[width=\textwidth, page=7]{chapters/chrgSTAR/img/selection/SDT.pdf}
	\end{subfigure}
	\begin{subfigure}{.45\textwidth}
	\includegraphics[width=\textwidth, page=3]{chapters/chrgSTAR/img/selection/SDT.pdf}
	\end{subfigure}
	\begin{subfigure}{.45\textwidth}
	\includegraphics[width=\textwidth, page=4]{chapters/chrgSTAR/img/selection/SDT.pdf}
	\end{subfigure}
	\begin{subfigure}{.45\textwidth}
	\includegraphics[width=\textwidth, page=9]{chapters/chrgSTAR/img/selection/SDT.pdf}
	\end{subfigure}
	\begin{minipage}{.45\textwidth}
		
		
		\caption[$N_{hits}^{fit}$ and $N_{hits}^{dE/dx}$, $DCA_{xy}$ and $|DCA_z|$, $d_0$ before applying the corresponding cuts]{Number of the  TPC hits used in the helix fit (top left) and number of the  TPC hits used to determine the $dE/dx$ (top right), the radial component (middle left) and the absolute magnitude of the longitudinal component (middle right) of the distance of the closest approach between  the global helix and the vertex, transverse impact parameter w.r.t. beam-line (bottom). All distributions are shown before applying  the corresponding cuts. Blue lines indicate regions accepted in the analysis.}
		\label{fig:dca_nhitsSTAR}
	\end{minipage}
\end{figure}
\captionsetup{format=default,indention=0pt,justification=justified}


\FloatBarrier