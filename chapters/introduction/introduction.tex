\chapter{Introduction}\label{chapter:introduction}
Inclusive measurements of charged-particle distributions in proton–proton ($pp$) collisions probe the strong interaction in the low-momentum transfer, non-perturbative regime of \ac{QCD}. In this kinematic region interactions are usually described by phenomenological models implemented in \ac{MC} event generators. Measurements can be used to constrain the free parameters of these models. An accurate description of low-energy strong interaction processes is essential for understanding and precise simulation of different types of $pp$ processes and the effects of multiple $pp$ collisions in the same bunch crossing at high instantaneous luminosity at hadron colliders. Measurements with tagging of the forward-scattered proton are of special interest. They give direct access to specific but still significant part of $pp$ processes called diffraction. In addition precise modelling of forward particle production is essential for better understanding of the longitudinal development of air showers observed in experiments studying cosmic radiation.

We present a measurement of charged particle production in events with single proton tagging (dominated by \ac{SD}: $p+p\to p+\textrm{X}$). The following observables are studied:
\begin{equation}
\frac{1}{N_\textrm{ev}}\frac{dN_\textrm{ev}}{dn_\textrm{ch}},\textrm{\hspace{1cm}} 
\frac{1}{N_\textrm{ev}}\frac{1}{2\pi p_\textrm{T}}\frac{d^2N}{d\bar{\eta}dp_\textrm{T}},\textrm{\hspace{1cm}} 
\frac{1}{N_\textrm{ev}}\frac{dN}{d\bar{\eta}}
\end{equation}
where $n_\textrm{ch}$ is the number of primary charged particles within kinematic range given by $p_\textrm{T}>200$~MeV and $|\eta|<0.7$, $N_\textrm{ev}$ is the 
total number of events with $2\leq n_\textrm{ch}\leq8$, $N$ is the total number of charged particles within the above kinematic acceptance and $\bar{\eta}$ is the pseudorapidity of the charged particle with longitudinal momentum taken with respect to direction of the forward scattered proton. To suppress non-SD events the trigger system required no signal in BBC-small in the direction of forward scattered proton and signal in BBC-small in opposite direction. The 
measurements are performed in a fiducial phase space of the forward scattered protons of $0.04<-t<0.16$~GeV$^2$/c$^2$ and 
$0.02<\xi<0.2$, where $\xi$ is the fractional energy loss of the scattered proton. In case of SD process $\xi=M^2_\textrm{X}/s$, where $M_\textrm{X}$ is 
the mass of the state $\textrm{X}$ into which one of the incoming proton dissociates and s is the center of mass energy squared of the $pp$ system. The above mentioned observables are presented in three $\xi$ 
regions: $0.02<\xi<0.05$, $0.05<\xi<0.1$ and $0.1<\xi<0.2$. In addition their average values in an event are presented as a function of $\xi$.

We have also studied an identified particle to antiparticle (pion, kaon, proton and their antiparticle) multiplicity ratios as a function of $p_\textrm{T}$ also in the above mentioned three regions of $\xi$. The system $\textrm{X}$ into which proton diffractively dissociates has net charge and baryon number $+1$. It is believed that initial charge and barion number should appear in the very forward direction leading to the equal amount of particles and antiparticles in the central region created by fragmentation and hadronization processes. However other scenarios are also possible where extra baryon is uniformly distributed over rapidity~\cite{Kopeliovich:1988qm} or even appear close to the gap edge~\cite{Bopp:2000vg}. It is natural to expect that possible charge and baryon number transfer to central region will be better visible at small $\xi$ where amount of particle-antiparticle creation is smaller due to the generally smaller particle multiplicity or due to the fact that gap edge is inside our fiducial region of $|\eta|<0.7$.
