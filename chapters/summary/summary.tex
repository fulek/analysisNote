\chapter{Summary and Conclusions}\label{chapter:summary}

Inclusive and identified (pion, kaon, proton and their antiparticles) charged-particle production in Single Diffractive process has been measured in proton-proton collisions at $\sqrt{s}= 200$ GeV with the STAR detector at RHIC using data corresponding to an integrated luminosity of 15 nb$^{-1}$.

Significant differences are observed between the measured distributions of $\xi$ and Monte Carlo model predictions. Among the models considered, EPOS and PYTHIA~8 (MBR) without suppression of diffractive cross sections at large $\xi$ provide the best description of the data.
Primary-charged-particle multiplicities and their densities as functions of pseudorapidity and transverse momentum are well described by PYTHIA~8 and EPOS SD$^\prime$ models. EPOS SD and HERWIG do not describe the data.
Similarity between the dissociation of a diffractivly produced system of mass $M_X$ and the hadronization of the system resulting from non-diffractive $pp$ collisions at $\sqrt{s} \approx M_X$ reported for the first time by UA4 Collaboration~\cite{ua4_diff1, UA4:intro, ua4_diff3} 
 was confirmed with much better precision.

$\pi^-/\pi^+$ and $K^-/K^+$ production ratios are close to unity and consistent with most of model predictions except for EPOS SD and HERWIG.
$\bar{p}/p$ production ratio shows a significant deviation from unity in the $0.02<\xi<0.05$ range indicating a non-negligible transfer of the baryon number from the forward to the central region. Equal amounts of protons and antiprotons are observed in the $\xi>0.05$ range. PYTHIA~8 and EPOS SD$^\prime$ agree with data for $\xi>0.05$.  For $0.02<\xi<0.05$ they predict small deviations from unity (0.93), however even larger effect is observed in the data ($0.86\pm 0.02)$. This observation is consistent with increase of the baryon number transfer to the central rapidity region with decreasing $\xi$ expected from~\cite{Bopp:2000vg} where an extra baryon can appear close to the rapidity gap edge (so called backward peak). 


At $p_{\textrm{T}}>0.5$ GeV measured $\left(K^-+K^+\right)/\left(\pi^-+\pi^+\right)$ ratio is significantly larger compared to inclusive inelastic measurements in $pp$ or $\bar{p}p$ collisions. This excess is not predicted by any model. This enhancement can be due to the high production rate of $gg\rightarrow \bar{s}s$ in the diffraction since Pomeron
is expected to be dominated by gluonic content.