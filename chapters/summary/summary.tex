\chapter{Summary and Conclusions}\label{chapter:summary}

Inclusive and identified (pion, kaon, proton and their antiparticles) charged particle production in Single Diffractive Dissociation process has been measured in proton-proton collisions at $\sqrt{s}= 200$ GeV with the STAR detector at RHIC using data corresponding to an integrated luminosity of 15 nb$^{-1}$.

Significant differences are observed between the measured distributions of $\xi$ and Monte Carlo model predictions. Amongst the models considered EPOS and PYTHIA8 (MBR) without suppression of diffractive cross sections at large $\xi$ provide the best description of the data.

Primary-charged-particle multiplicity and its density as a function of pseudorapidity and transverse momentum are well described by PYTHIA8 and EPOS-SD$^\prime$ models.  EPOS-SD and HERWIG do not describe the data.

$\pi^-/\pi^+$ and $K^-/K^+$ production ratios are close to unity and consistent with most of model predictions except for EPOS-SD and HERWIG.

$\bar{p}/p$ production ratio shows a significant deviation from unity 
in the $0.02<\xi<0.05$ range indicating a non-negligible transfer of the baryon number from the forward to the central region. Equal amount of protons and antiprotons are observed in the $\xi>0.05$ range.
PYTHIA8 and EPOS-SD$^\prime$ agree with data for $\xi>0.05$. 
For $0.02<\xi<0.05$ they predict small deviations from unity (0.93) which is however higher than observed in data ($0.86\pm 0.02)$. 
HERWIG and EPOS-SD predict much larger baryon number transfers compared to data for $\xi<0.1$ and show consistency with data for $\xi>0.1$.